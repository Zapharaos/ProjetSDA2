\pdfbookmark[0]{Abstract}{Abstract}
\chapter*{Problème}
\label{sec:abstract}
\vspace*{-10mm}

On se propose d’étudier dans ce projet une méthode de détection automatique d’une langue.
Pour simplifier cette étude, seuls l’allemand, l’anglais ainsi que le français seront considérés. Les
accents et la ponctuation que l’on peut retrouver dans la langue française et allemande ne seront
quant à eux pas utilisés.

\section*{\color{black}Approche proposée}
\begin{enumerate}
\item Analyse des dictionnaires allemand, anglais et français.
\item Lecture du texte à identifier.
\item Pour chaque mot du texte à identifier, tester la correspondance du mot par rapport à
chacun des dictionnaires
\item La langue du texte est celle avec le plus grand nombre de correspondances positives.
\end{enumerate}

\newline Pour ce faire, on nous propose deux structures de données que nous allons devoir implémenter : le trie et le dawg.\newline Ces structures vont contenir les dictionnaires allemand, anglais et français.

